\documentclass[10pt]{scrartcl}
\usepackage[utf8]{inputenc}
\usepackage[T1]{fontenc}
\usepackage{fancyhdr}
\usepackage{graphicx}
\usepackage{eurosym}
\usepackage{amsmath}
\usepackage[ngerman]{babel}
\date{\vspace{-5ex}}
\usepackage{geometry}
\geometry{a4paper, left=25mm, right=25mm}
\renewcommand{\headrulewidth}{0pt}
\fancyhead[L]{}
\fancyhead[R]{
\includegraphics[width=4cm]{images/shack.png}
}
\input{signature_head}

\pagestyle{plain}
\title{\flushleft{FESTOOL Bandschleifer BS 120}}
\begin{document}
\maketitle
\thispagestyle{fancy}
\section{Schutzkleidung}
\begin{itemize}
\item \textbf{keine Handschuhe}
\item Schutzbrille
\item Gehörschutz
\item ggf. Haare zurückbinden 
\item ggf. Atemschutz 
\end{itemize}
\section{Allgemeine Sicherheitshinweise}
\begin{itemize}
\item \textbf{Bei Arbeiten mit Metall keine Absaugung verwenden! Brandgefahr!}
\item Keine defekten, rissigen oder anderweitig beschädigten Schleifbänder verwenden 
\item Vor Inbetriebnahme überprüfen dass der Bandlauf ordentlich eingestellt ist
\item Sicherstellen dass keine Bauteile wie Anschläge, Abdeckungen, usw. am Band streifen
\end{itemize}

\section{Inbetriebnahme}
\begin{enumerate}
\item Modul in CMS Tisch einsetzen
\item Modul verriegeln
\item Strom an CMS anschliessen
\item Netzstecker des CMS an Staubsauger anschliessen
\item Staubsauger auf MAN stellen
\item Staubsauger einstecken
\end{enumerate}

\section{Aufräumen}
\begin{enumerate}
\item Modul reinigen
\item Modul wieder in den Schrank räumen
\end{enumerate}

\section{Technische Daten}
Leistungsaufnahme: 550W\\
Leerlaufdrehzahl: 2800/min\\
Bandlänge: 820mm\\
Bandbreite: 120mm\\
Schleifbreite: 120mm\\
Schleiffläche: 180mm\\

\input{signature_body}

\end{document}
