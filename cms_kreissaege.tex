\documentclass[10pt]{scrartcl}
\usepackage[utf8]{inputenc}
\usepackage[T1]{fontenc}
\usepackage{fancyhdr}
\usepackage{graphicx}
\usepackage{eurosym}
\usepackage{amsmath}
\usepackage[ngerman]{babel}
\date{\vspace{-5ex}}
\usepackage{geometry}
\geometry{a4paper, left=25mm, right=25mm}
\renewcommand{\headrulewidth}{0pt}
\fancyhead[L]{}
\fancyhead[R]{
\includegraphics[width=4cm]{images/shack.png}
}
\include{signature_head}
\pagestyle{plain}
\title{\flushleft{FESTOOL Kreiss\"age CMS 75}}
\begin{document}
\maketitle
\thispagestyle{fancy}
\section{Schutzkleidung}
\begin{itemize}
\item \textbf{keine Handschuhe}
\item Schutzbrille
\item Gehörschutz
\item ggf. Haare zurückbinden 
\end{itemize}
\section{Allgemeine Sicherheitshinweise}
\begin{itemize}
\item \textbf{Bei Arbeiten mit Metall keine Absaugung verwenden! Brandgefahr!}
\item Hände bei aktiver Säge immer vom Sägebereich und dem Sägeblatt fernhalten.

\item Nicht unter das Werkstück greifen. Die Schutzhaube kann unterhalb des Werkstückes nicht vor dem Sägeblatt schützen.

\item Die Schnitttiefe an die Dicke des Werkstücks anpassen. Es sollte weniger als eine volle Zahnhöhe unter dem Werkstück sichtbar sein.

\item Das zu sägende Werkstück niemals in der Hand oder über dem Bein halten. Das Werkstück stabil und sicher fixieren. Es ist wichtig, das Werkstück gut zu befestigen,
um die Gefahr von Körperkontakt, Klemmen des Sägeblattes oder Verlust der Kontrolle zu minimieren.

\item Auf keinen Fall so sägen, dass das Stromkabel getroffen werden kann. 

\item Beim Schneiden immer einen Anschlag oder eine gerade Kantenführung verwenden. Dies verbessert die Schnittgenauigkeit und verringert die Möglichkeit, dass das Sägeblatt klemmt.

\item Nur Sägeblätter in der richtigen Größe und mit passender Aufnahmebohrung (z.B. sternförmig oder rund) verwenden. Sägeblätter, die nicht zu den Montageteilen der Säge passen, laufen unrund und führen zum Verlust der Kontrolle oder beschädigen die Säge.

\item Niemals beschädigte oder falsche Sägeblatt-Spannflansche oder -Schrauben verwenden. Die Sägeblatt-Spannflansche und Schrauben wurden speziell für diese Säge konstruiert, für optimale Leistung und Betriebssicherheit.

\end{itemize}

\section{Inbetriebnahme}
\begin{enumerate}
\item Modul in CMS Tisch einsetzen
\item Modul verriegeln
\item Strom an CMS anschliessen
\item Schalter fixieren
\item Netzstecker des CMS an Staubsauger anschliessen
\item Staubsauger auf MAN stellen
\item Staubsauger einstecken
\item Sägeblatt herausfahren
\end{enumerate}

\section{Aufräumen}
\begin{enumerate}
\item Modul reinigen
\item Fixierung des Schalters entfernen
\item Modul wieder in den Schrank räumen
\end{enumerate}


\section{Werkstoffe}
Mit dem Standard Sägeblatt dürfen folgende Materialien bearbeitet werden:
\begin{itemize}
\item alle Holzwerkstoffe
\item Baustoffplatten
\item weiche Kunststoffe
\end{itemize}

\section{Technische Daten}
Leistungsaufnahme: 1600W\\
Leerlaufdrehzahl: 1350-3550/min\\
Bohrungs-Ø: 30mm\\
Sägeblatt-Ø: 210mm\\
Schnitthöhe: 0-70mm\\
Schnitthöhe bei 45 Grad: 0-48mm\\

\include{signature_body}

\end{document}
